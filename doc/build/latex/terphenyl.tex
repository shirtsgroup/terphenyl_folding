%% Generated by Sphinx.
\def\sphinxdocclass{report}
\documentclass[letterpaper,12pt,english,openany,oneside]{sphinxmanual}
\ifdefined\pdfpxdimen
   \let\sphinxpxdimen\pdfpxdimen\else\newdimen\sphinxpxdimen
\fi \sphinxpxdimen=.75bp\relax

\PassOptionsToPackage{warn}{textcomp}
\usepackage[utf8]{inputenc}
\ifdefined\DeclareUnicodeCharacter
% support both utf8 and utf8x syntaxes
  \ifdefined\DeclareUnicodeCharacterAsOptional
    \def\sphinxDUC#1{\DeclareUnicodeCharacter{"#1}}
  \else
    \let\sphinxDUC\DeclareUnicodeCharacter
  \fi
  \sphinxDUC{00A0}{\nobreakspace}
  \sphinxDUC{2500}{\sphinxunichar{2500}}
  \sphinxDUC{2502}{\sphinxunichar{2502}}
  \sphinxDUC{2514}{\sphinxunichar{2514}}
  \sphinxDUC{251C}{\sphinxunichar{251C}}
  \sphinxDUC{2572}{\textbackslash}
\fi
\usepackage{cmap}
\usepackage[T1]{fontenc}
\usepackage{amsmath,amssymb,amstext}
\usepackage{babel}



\usepackage{times}
\expandafter\ifx\csname T@LGR\endcsname\relax
\else
% LGR was declared as font encoding
  \substitutefont{LGR}{\rmdefault}{cmr}
  \substitutefont{LGR}{\sfdefault}{cmss}
  \substitutefont{LGR}{\ttdefault}{cmtt}
\fi
\expandafter\ifx\csname T@X2\endcsname\relax
  \expandafter\ifx\csname T@T2A\endcsname\relax
  \else
  % T2A was declared as font encoding
    \substitutefont{T2A}{\rmdefault}{cmr}
    \substitutefont{T2A}{\sfdefault}{cmss}
    \substitutefont{T2A}{\ttdefault}{cmtt}
  \fi
\else
% X2 was declared as font encoding
  \substitutefont{X2}{\rmdefault}{cmr}
  \substitutefont{X2}{\sfdefault}{cmss}
  \substitutefont{X2}{\ttdefault}{cmtt}
\fi


\usepackage[Bjarne]{fncychap}
\usepackage{sphinx}

\fvset{fontsize=\small}
\usepackage{geometry}

% Include hyperref last.
\usepackage{hyperref}
% Fix anchor placement for figures with captions.
\usepackage{hypcap}% it must be loaded after hyperref.
% Set up styles of URL: it should be placed after hyperref.
\urlstyle{same}

\usepackage{sphinxmessages}
\setcounter{tocdepth}{1}



\title{Terphenyl folding Documentation}
\date{Sep 22, 2019}
\release{0.0}
\author{Garrett A. Meek\\Theodore L. Fobe\\Benjamin J. Coscia\\ \\Research group of Professor Michael R. Shirts\\ \\Dept. of Chemical and Biological Engineering\\University of Colorado Boulder}
\newcommand{\sphinxlogo}{\vbox{}}
\renewcommand{\releasename}{Release}
\makeindex
\begin{document}

\pagestyle{empty}
\sphinxmaketitle
\pagestyle{plain}
\sphinxtableofcontents
\pagestyle{normal}
\phantomsection\label{\detokenize{index::doc}}


This documentation is generated automatically using Sphinx, which reads all docstring-formatted comments from Python functions in the ‘terphenyl\_folding’ repository.  (See terphenyl\_folding/doc for Sphinx source files.)


\chapter{Terphenyl folding simulation utilities}
\label{\detokenize{simulation:terphenyl-folding-simulation-utilities}}\label{\detokenize{simulation::doc}}
Shown below are tools that allow simulation of terphenyl oligomers.

\phantomsection\label{\detokenize{simulation:module-simulation}}\index{simulation (module)@\spxentry{simulation}\spxextra{module}}\index{build\_directories() (in module simulation)@\spxentry{build\_directories()}\spxextra{in module simulation}}

\begin{fulllineitems}
\phantomsection\label{\detokenize{simulation:simulation.build_directories}}\pysiglinewithargsret{\sphinxcode{\sphinxupquote{simulation.}}\sphinxbfcode{\sphinxupquote{build\_directories}}}{\emph{polymer\_name}, \emph{polymer\_length}, \emph{fresh\_run=False}}{}
Given a set of input strings, this function builds the directories that are needed to perform GROMACS simulations with terphenyl oligomers.
\begin{quote}\begin{description}
\item[{Parameters}] \leavevmode\begin{itemize}
\item {} 
\sphinxstyleliteralstrong{\sphinxupquote{polymer\_name}} (\sphinxhref{https://docs.python.org/3/library/stdtypes.html\#str}{\sphinxstyleliteralemphasis{\sphinxupquote{str}}}) \textendash{} The name of the polymer

\item {} 
\sphinxstyleliteralstrong{\sphinxupquote{polymer\_length}} (\sphinxhref{https://docs.python.org/3/library/stdtypes.html\#str}{\sphinxstyleliteralemphasis{\sphinxupquote{str}}}) \textendash{} The length of the polymer we are modeling (in monomer units)

\item {} 
\sphinxstyleliteralstrong{\sphinxupquote{run\_directory}} (\sphinxhref{https://docs.python.org/3/library/stdtypes.html\#str}{\sphinxstyleliteralemphasis{\sphinxupquote{str}}}) \textendash{} The directory where simulations will be run

\item {} 
\sphinxstyleliteralstrong{\sphinxupquote{fresh\_run}} (\sphinxstyleliteralemphasis{\sphinxupquote{Logical}}) \textendash{} A logical variable determining whether old run files should be removed.

\end{itemize}

\item[{Returns}] \leavevmode
\begin{itemize}
\item {} 
run\_directory ( str ) - The path to a directory where simulations will be run.

\item {} 
pdb\_file ( str ) - The path to pdb file that will be used for simulations.

\item {} 
solvent\_file ( str ) - The path to a file containing a box of solvent molecules

\item {} 
topology\_file ( str ) - The path to a file containing the topology for the pdb\_file.

\end{itemize}


\end{description}\end{quote}

\end{fulllineitems}

\index{compress\_large\_files() (in module simulation)@\spxentry{compress\_large\_files()}\spxextra{in module simulation}}

\begin{fulllineitems}
\phantomsection\label{\detokenize{simulation:simulation.compress_large_files}}\pysiglinewithargsret{\sphinxcode{\sphinxupquote{simulation.}}\sphinxbfcode{\sphinxupquote{compress\_large\_files}}}{\emph{directory}, \emph{size\_threshold=100000000.0}}{}
\end{fulllineitems}

\index{equilibrate() (in module simulation)@\spxentry{equilibrate()}\spxextra{in module simulation}}

\begin{fulllineitems}
\phantomsection\label{\detokenize{simulation:simulation.equilibrate}}\pysiglinewithargsret{\sphinxcode{\sphinxupquote{simulation.}}\sphinxbfcode{\sphinxupquote{equilibrate}}}{}{}
\end{fulllineitems}

\index{minimize() (in module simulation)@\spxentry{minimize()}\spxextra{in module simulation}}

\begin{fulllineitems}
\phantomsection\label{\detokenize{simulation:simulation.minimize}}\pysiglinewithargsret{\sphinxcode{\sphinxupquote{simulation.}}\sphinxbfcode{\sphinxupquote{minimize}}}{\emph{mdrun\_file}, \emph{topology}, \emph{input\_structure}}{}
\end{fulllineitems}

\index{parameterize() (in module simulation)@\spxentry{parameterize()}\spxextra{in module simulation}}

\begin{fulllineitems}
\phantomsection\label{\detokenize{simulation:simulation.parameterize}}\pysiglinewithargsret{\sphinxcode{\sphinxupquote{simulation.}}\sphinxbfcode{\sphinxupquote{parameterize}}}{\emph{param\_directory}, \emph{pdb\_file}, \emph{topology\_file}}{}
Given a directory path, PDB file, and a topology file, this function parameterizes the structure with GAFF.
\begin{quote}\begin{description}
\item[{Parameters}] \leavevmode\begin{itemize}
\item {} 
\sphinxstyleliteralstrong{\sphinxupquote{param\_directory}} (\sphinxhref{https://docs.python.org/3/library/stdtypes.html\#str}{\sphinxstyleliteralemphasis{\sphinxupquote{str}}}) \textendash{} The path to a directory where intermediate and output parameterization files will be written.

\item {} 
\sphinxstyleliteralstrong{\sphinxupquote{pdb\_file}} (\sphinxhref{https://docs.python.org/3/library/stdtypes.html\#str}{\sphinxstyleliteralemphasis{\sphinxupquote{str}}}) \textendash{} The path to a PDB file containing data for the structure that will be parameterized.

\item {} 
\sphinxstyleliteralstrong{\sphinxupquote{topology\_file}} (\sphinxhref{https://docs.python.org/3/library/stdtypes.html\#str}{\sphinxstyleliteralemphasis{\sphinxupquote{str}}}) \textendash{} The path to a file containing the topology for the structure that will be parameterized.

\end{itemize}

\end{description}\end{quote}

\end{fulllineitems}

\index{replace() (in module simulation)@\spxentry{replace()}\spxextra{in module simulation}}

\begin{fulllineitems}
\phantomsection\label{\detokenize{simulation:simulation.replace}}\pysiglinewithargsret{\sphinxcode{\sphinxupquote{simulation.}}\sphinxbfcode{\sphinxupquote{replace}}}{\emph{file}, \emph{original\_text}, \emph{replacement\_text}}{}
Given a file, a target search string, and a replacement string, this function replaces the text in ‘file’.
\begin{quote}\begin{description}
\item[{Parameters}] \leavevmode\begin{itemize}
\item {} 
\sphinxstyleliteralstrong{\sphinxupquote{file}} (\sphinxstyleliteralemphasis{\sphinxupquote{file}}) \textendash{} A file containing the text that will be replaced.

\item {} 
\sphinxstyleliteralstrong{\sphinxupquote{original\_text}} (\sphinxhref{https://docs.python.org/3/library/stdtypes.html\#str}{\sphinxstyleliteralemphasis{\sphinxupquote{str}}}) \textendash{} Text that will be replaced.

\item {} 
\sphinxstyleliteralstrong{\sphinxupquote{replacement\_text}} (\sphinxhref{https://docs.python.org/3/library/stdtypes.html\#str}{\sphinxstyleliteralemphasis{\sphinxupquote{str}}}) \textendash{} Text that will be used to replace the original text.

\end{itemize}

\end{description}\end{quote}

\end{fulllineitems}

\index{simulate() (in module simulation)@\spxentry{simulate()}\spxextra{in module simulation}}

\begin{fulllineitems}
\phantomsection\label{\detokenize{simulation:simulation.simulate}}\pysiglinewithargsret{\sphinxcode{\sphinxupquote{simulation.}}\sphinxbfcode{\sphinxupquote{simulate}}}{}{}
\end{fulllineitems}

\index{solvate() (in module simulation)@\spxentry{solvate()}\spxextra{in module simulation}}

\begin{fulllineitems}
\phantomsection\label{\detokenize{simulation:simulation.solvate}}\pysiglinewithargsret{\sphinxcode{\sphinxupquote{simulation.}}\sphinxbfcode{\sphinxupquote{solvate}}}{\emph{polymer\_name}, \emph{polymer\_length}, \emph{polymer\_code}, \emph{input\_pdb}, \emph{run\_directory}, \emph{solvent\_density=0.5}}{}
\end{fulllineitems}



\chapter{Terphenyl folding analysis tools}
\label{\detokenize{analysis:terphenyl-folding-analysis-tools}}\label{\detokenize{analysis::doc}}
Shown below are functions/tools that allow analysis of terphenyl oligomer simulation results.

\phantomsection\label{\detokenize{analysis:module-analysis}}\index{analysis (module)@\spxentry{analysis}\spxextra{module}}\index{construct\_selector() (in module analysis)@\spxentry{construct\_selector()}\spxextra{in module analysis}}

\begin{fulllineitems}
\phantomsection\label{\detokenize{analysis:analysis.construct_selector}}\pysiglinewithargsret{\sphinxcode{\sphinxupquote{analysis.}}\sphinxbfcode{\sphinxupquote{construct\_selector}}}{\emph{atom\_list}, \emph{res\_dict}, \emph{n\_residues}}{}
Since our molecule is all in one residue, we need some interesting ways to extract residues, instead of a simple
select string
\begin{description}
\item[{atom\_list}] \leavevmode{[}list{]}
String name of base atoms in the first residue

\item[{res\_dict}] \leavevmode{[}dict{]}
Dictionary telling how many atoms of a given element are in each residue

\end{description}

\end{fulllineitems}

\index{get\_internal\_coordinate\_definitions() (in module analysis)@\spxentry{get\_internal\_coordinate\_definitions()}\spxextra{in module analysis}}

\begin{fulllineitems}
\phantomsection\label{\detokenize{analysis:analysis.get_internal_coordinate_definitions}}\pysiglinewithargsret{\sphinxcode{\sphinxupquote{analysis.}}\sphinxbfcode{\sphinxupquote{get\_internal\_coordinate\_definitions}}}{\emph{structure}, \emph{polymer\_name}, \emph{polymer\_length}}{}
structure: pdb file path

polymer\_name: string (ie: “o-terphenyl”)

polymer\_length: string (ie: “monomer”)

\end{fulllineitems}

\index{get\_phenyl\_carbon\_indices() (in module analysis)@\spxentry{get\_phenyl\_carbon\_indices()}\spxextra{in module analysis}}

\begin{fulllineitems}
\phantomsection\label{\detokenize{analysis:analysis.get_phenyl_carbon_indices}}\pysiglinewithargsret{\sphinxcode{\sphinxupquote{analysis.}}\sphinxbfcode{\sphinxupquote{get\_phenyl\_carbon\_indices}}}{\emph{polymer\_name}, \emph{polymer\_length}}{}
polymer\_name: string (ie: “o-terphenyl”)

polymer\_length: string (ie: “monomer”)

phenyl\_indices\_list: List( dict(‘1’: {[}phenyl-1 carbon indices{]}, ‘2’: {[}phenyl-2 carbon indices{]}, ‘3’: {[}phenyl-3 carbon indices{]}))

\end{fulllineitems}

\index{get\_phenyl\_centers\_of\_mass() (in module analysis)@\spxentry{get\_phenyl\_centers\_of\_mass()}\spxextra{in module analysis}}

\begin{fulllineitems}
\phantomsection\label{\detokenize{analysis:analysis.get_phenyl_centers_of_mass}}\pysiglinewithargsret{\sphinxcode{\sphinxupquote{analysis.}}\sphinxbfcode{\sphinxupquote{get\_phenyl\_centers\_of\_mass}}}{\emph{structure}, \emph{polymer\_name}, \emph{polymer\_length}}{}
structure: pdb file path

polymer\_name: string (ie: “o-terphenyl”)

polymer\_length: string (ie: “monomer”)

\end{fulllineitems}

\index{read\_trajectory() (in module analysis)@\spxentry{read\_trajectory()}\spxextra{in module analysis}}

\begin{fulllineitems}
\phantomsection\label{\detokenize{analysis:analysis.read_trajectory}}\pysiglinewithargsret{\sphinxcode{\sphinxupquote{analysis.}}\sphinxbfcode{\sphinxupquote{read\_trajectory}}}{\emph{structure}, \emph{trajectory}}{}
structure: pdb file path

trajectory: xtc file path

\end{fulllineitems}



\renewcommand{\indexname}{Python Module Index}
\begin{sphinxtheindex}
\let\bigletter\sphinxstyleindexlettergroup
\bigletter{a}
\item\relax\sphinxstyleindexentry{analysis}\sphinxstyleindexpageref{analysis:\detokenize{module-analysis}}
\indexspace
\bigletter{s}
\item\relax\sphinxstyleindexentry{simulation}\sphinxstyleindexpageref{simulation:\detokenize{module-simulation}}
\end{sphinxtheindex}

\renewcommand{\indexname}{Index}
\printindex
\end{document}